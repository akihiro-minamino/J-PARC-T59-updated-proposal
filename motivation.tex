\section{Motivation}
The T2K (Tokai-to-Kamioka) experiment is a long baseline neutrino oscillation experiment.
In 2013, T2K made the first observation of electron neutrino appearance in a muon neutrino beam
with a $7.3\sigma$ significance and constrained the CP violating phase $\delta_{CP}$ based on
a data set corresponding to $8.4\%$ of the approved delivered Protons On Target (POT) \cite{t2k_nue_appearance_2013}.
In 2016, T2K indicated the CP violation with $90\%$ confidence level based on
a data set corresponding to $20\%$ of the approved delivered POT\cite{t2k_cp_2016}.
T2K is increasing its POT toward for the discovery of the CP violation.


T2K uses Super-Kamiokande (SK) as the far detector to measure neutrino interactions
at a distance of 295 km from the accelerator, and near detectors at J-PARC.
The near detectors consist of an on-axis Interactive Neutrino Grid detector (INGRID)
and an off-axis detector, ND280.
Uncertainties on neutrino flux and cross sections for T2K oscillation analyses are largely constrained
by ND280 measurement.
However, some part of systematic errors for neutrino cross sections cannot be fully constrained by the ND280 measurement because of its limited angular acceptance for the charged particle tracks.
The uncertainty (rms/mean in \%) on the predicted number of signal $\nu_{e}$ events for each group
of systematic uncertainties in the T2K oscillation analysis \cite{t2K_cp_2016} is shown in Table \ref{table:t2k_error_2016}.
In order to fully exploit the statistical power, there is an urgent need to reduce the neutrino cross section errors.


In the test experiment, we will develop a new neutrino detector to measure neutrino cross sections
on water and hydrocarbon targets with high precision and large angular acceptance.
A new idea, a 3D grid-like structure of scintillator bars, is adopted to detect tracks of charged particles
with $4\pi$ angular acceptance and larger mass ratio of water to scintillator bars.
The goal of the test experiment is to test the basic performance of the detector,
such as track reconstruction efficiency and particle identification capability using neutrino beam data,
and confirm the capability of measuring the cross section.
In order to achieve the goal, we would like to use the B2 floor of the near detector hall of J-PARC neutrino beam-line as a test facility of the neutrino beam.
We request $1 \times 10^{21} POT$ of $\nu$ (not anti-$\nu$) beam with the full detector configuration.
